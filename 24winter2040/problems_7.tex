\documentclass[a4paper,11pt]{article}


\usepackage{multicol}
\usepackage[utf8]{inputenc}
\usepackage[T1]{fontenc}
%\usepackage[portuguese]{babel}

%\usepackage[osf,sc]{mathpazo}
\usepackage[top=1in, bottom=1in, left=1.25in, right=1.25in]{geometry}

\usepackage{amsmath,amsthm,amssymb}
\usepackage{mathrsfs} % \mathsf

\usepackage[demo]{graphicx}
\usepackage[shortlabels]{enumitem}
\usepackage{tikz-cd}

\usepackage{quiver}
% surrounding boxes
\usepackage{tcolorbox}
\tcbuselibrary{breakable,skins}

\usepackage{hyperref}
\hypersetup{
    colorlinks,
    linkcolor={red!50!black},
    citecolor={blue!50!black},
    urlcolor={blue!80!black}
}

\title{Problem Set 7}
\author{Matrix Theory \& Linear Algebra II}
\date{Winter 2025}
% --new title page--



% \swapnumbers % 1.1.1. Teorema

\newtheorem{theorem}{Theorem}
\newtheorem{axiom}[theorem]{Axiom}
\newtheorem{claim}[theorem]{Claim}
\newtheorem{corollary}[theorem]{Corollary}
\newtheorem{lemma}[theorem]{Lemma}
\newtheorem{proposition}[theorem]{Proposition}


\theoremstyle{definition}
\newtheorem{remark}[theorem]{Remark}
\newtheorem{definition}[theorem]{Definition}
\newtheorem{example}[theorem]{Example}
\newtheorem{exercise}[theorem]{Exercício}
\newtheorem{fact}[theorem]{Fato}
\newtheorem{notation}[theorem]{Notation}



\newcommand{\bC}{\mathcal C}
\newcommand{\bF}{\mathcal F}
\newcommand{\bR}{\mathcal R}

\newcommand{\cP}{\mathcal P}

\newcommand{\dashline}{\noindent\rule{2cm}{0.4pt}}

\begin{document}
\maketitle
\thispagestyle{empty}


\noindent
In this problem set, any vector space $V$ comes equipped with an inner product.
The field $\mathbb F$ is either $\mathbb C$ or $\mathbb R$, unless specified.
Ensure you do at least the first page of this document.

\begin{enumerate}[(1)]
    \item 
    True or false:
    \begin{enumerate}
    \item Every unitary operator $U : X \to X$ is normal.
    \item A matrix is unitary if and only if it is invertible.
    \item If two matrices are unitarily equivalent, then they are also similar.
    \item The sum of self-adjoint operators is self-adjoint.
    \item The adjoint of a unitary operator is unitary.
    \item The adjoint of a normal operator is normal.
    \item If all eigenvalues of a linear operator are 1, then the operator must be unitary or orthogonal.
    \item If all eigenvalues of a normal operator are 1, then the operator is identity.
    \item A linear operator may preserve norm but not the inner product.
    \end{enumerate}

    
    \item 
    Four of the following matrices are diagonalizable. Which ones and why?
    \begin{multicols}{2}
    \begin{enumerate}
    \item[(a)]
    \[
    \begin{bmatrix}
    0 & -1 \\
    1 & 0
    \end{bmatrix}
    \]
    
    \item[(b)]
    \[
    \begin{bmatrix}
    2 & i & 0 \\
    -i & 3 & 4 \\
    0 & 4 & -1
    \end{bmatrix}
    \]
    
    \item[(c)] 
    \[
    \begin{bmatrix}
    1+i & 2 & -i \\
    0 & 3-i & 4 \\
    0 & 0 & 5+2i
    \end{bmatrix}
    \]
    \end{enumerate}
    \columnbreak
    \begin{enumerate}
    \item[(d)]
    \[
    \begin{bmatrix}
    0.5 & -2 & 3 & 1 \\
    -1 & 4.2 & 0 & 3.5 \\
    2 & -0.5 & 1.3 & 2.2 \\
    4 & -3 & 2 & -1
    \end{bmatrix}
    \]
    
    \item[(e)]
    \[
    \begin{bmatrix}
    2 & 1 & 0 & -1 & 3 \\
    1 & 4 & 2 & 0 & -2 \\
    0 & 2 & 5 & 1 & 4 \\
    -1 & 0 & 1 & 3 & 2 \\
    3 & -2 & 4 & 2 & 6
    \end{bmatrix}
    \]
    \end{enumerate}
    \end{multicols}
    
    \item
    Check that the following real matrices are orthogonal and/or self-adjoint, and orthogonally diagonalize them.
    In other words, find orthonormal bases of eigenvectors in each case.
    \[
    A=\begin{bmatrix}
        0&1 \\ 1&0
    \end{bmatrix},
    \quad
    B=\dfrac{1}{2}\begin{bmatrix}
        1&-\sqrt 3 \\ \sqrt 3&1
    \end{bmatrix},
    \quad
    C=\dfrac{\sqrt 2}{2}\begin{bmatrix}
        1&1\\ 1&1
    \end{bmatrix}.
    \]
    In each case give a geometric interpretation of the transformation.\newpage
    \item 
    Prove the following properties of adjoint operators.
    You can do this from the definition, or by checking properties on a basis.
    \begin{enumerate}[(a)]
        \item 
        $(S+T)* = S^*+T*$.
        \item 
        $(\lambda\cdot T)^*=\overline\lambda\cdot T^*$.
        \item
        $(T^*)^* = T$
        \item
        $(ST)^*=T^*S^*$
        \item 
        if $T$ is invertible, then $(T^{-1})^*=(T^*)^{-1}$
        \item 
        if $\lambda$ is an eigenvalue of $T$, then $\overline\lambda$ is an eigenvalue of $T^*$.
    \end{enumerate}

    \item 
    Show that $\ker T = (\text{range}\,T^*)^\perp$.

    \item 
    Let $T:V\to V$ be a self-adjoint operator.
    Show that if $\lambda_1\neq \lambda_2$ are distinct eigenvalues, then the corresponding eigenvectors are orthogonal.
    Use this and the Spectral Theorem to conclude that a self-adjoint operator has an orthonormal basis of eigenvectors.

    \item 
    An operator $T:V\to W$ is an \textit{isometry} if $\langle Tv,Tw\rangle = \langle v,w\rangle$ for all $v,w\in V$.
    Show that if $\dim V=\dim W$ then any isometry is invertible.
    
    
    \item 
    An invertible operator $T:V\to V$ is \textit{unitary} if it is an isometry (in particular, $T$ is invertible).
    Show that an operator is unitary if and only if $TT^*=T^*T=I$ (i.e. $U^{-1}=U^*$).
    
    \item 
    Using the previous exercise, explain the following assertion: ``Unitary and orthogonal operators are the operators that preserve angles and distances.''

    \item Let \( U \) be a \( 2 \times 2 \) orthogonal matrix with \( \det U = 1 \). Prove that \( U \) is a rotation matrix.

    \item
    Let \( U \) be a \( 3 \times 3 \) orthogonal matrix with \( \det U = 1 \). Prove that 

\begin{itemize}
    \item[(a)] \( 1 \) is an eigenvalue of \( U \).
    \item[(b)] If \( \mathbf{v}_1, \mathbf{v}_2, \mathbf{v}_3 \) is an orthonormal basis, such that \( U \mathbf{v}_1 = \mathbf{v}_1 \) (remember, that \( 1 \) is an eigenvalue), then in this basis the matrix of \( U \) is
    \[
    \begin{pmatrix}
    1 & 0 & 0 \\
    0 & \cos \alpha & -\sin \alpha \\
    0 & \sin \alpha & \cos \alpha
    \end{pmatrix},
    \]
    where \( \alpha \) is some angle.
\end{itemize}

    \item 
    Let $A$ be an $m\times n$ matrix.
    Show that
    \begin{enumerate}
        \item
        $A^*A$ is self-adjoint.
        \item
        The eigenvalues of $A^*A$ are non-negative.
        \item 
        $A^*A+I$ is invertible.
    \end{enumerate}

    \item 
    Prove that a normal operator with whose eigenvalues satisfy $\|\lambda_k\| = 1$ is unitary. Hint: diagonalize.

    \item 
    For $\mathbb F = \mathbb R$, show that self-adjoint operators form a subspace of $\mathcal L(V)$.(\textit{Hint:} this was in your 2nd WebWork homework.)
    Show that this is false for $\mathbb F=\mathbb C$.

\end{enumerate}

\end{document}