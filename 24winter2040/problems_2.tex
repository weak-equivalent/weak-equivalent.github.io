\documentclass[a4paper,11pt]{article}



\usepackage[utf8]{inputenc}
\usepackage[T1]{fontenc}
%\usepackage[portuguese]{babel}

%\usepackage[osf,sc]{mathpazo}
\usepackage[top=1in, bottom=1in, left=1.25in, right=1.25in]{geometry}

\usepackage{amsmath,amsthm,amssymb}
\usepackage{mathrsfs} % \mathsf

\usepackage[demo]{graphicx}
\usepackage[shortlabels]{enumitem}
\usepackage{tikz-cd}

\usepackage{quiver}
% surrounding boxes
\usepackage{tcolorbox}
\tcbuselibrary{breakable,skins}

\usepackage{hyperref}
\hypersetup{
    colorlinks,
    linkcolor={red!50!black},
    citecolor={blue!50!black},
    urlcolor={blue!80!black}
}

\title{Problem Set 2}
\author{Matrix Theory \& Linear Algebra II}
\date{}
% --new title page--


\setlength{\abovedisplayskip}{1pt}
\setlength{\belowdisplayskip}{1pt}

% \swapnumbers % 1.1.1. Teorema

\newtheorem{theorem}{Theorem}
\newtheorem{axiom}[theorem]{Axiom}
\newtheorem{claim}[theorem]{Claim}
\newtheorem{corollary}[theorem]{Corollary}
\newtheorem{lemma}[theorem]{Lemma}
\newtheorem{proposition}[theorem]{Proposition}


\theoremstyle{definition}
\newtheorem{remark}[theorem]{Remark}
\newtheorem{definition}[theorem]{Definition}
\newtheorem{example}[theorem]{Example}
\newtheorem{exercise}[theorem]{Exercício}
\newtheorem{fact}[theorem]{Fato}
\newtheorem{notation}[theorem]{Notation}



\newcommand{\bC}{\mathcal C}
\newcommand{\bF}{\mathcal F}
\newcommand{\bR}{\mathcal R}

\newcommand{\cP}{\mathcal P}

\newcommand{\dashline}{\noindent\rule{2cm}{0.4pt}}

\begin{document}
\maketitle
\thispagestyle{empty}

\noindent
In this problem set, $\mathbb F$ denotes either $\mathbb R$ or $\mathbb C$ (i.e. if $\mathbb F$ is in the question, then the solution should be agnostic to whether it was $\mathbb R$ or $\mathbb C$) and $V$ denotes a vector space over $\mathbb F$.

\begin{enumerate}[(1)]
\item
Which of the following lists are linearly independent?
\begin{enumerate}[(a)]
    \item 
    $
    \mathbf{v}_1 = \begin{pmatrix} 1 \\ 2 \\ 3 \end{pmatrix}, \quad
    \mathbf{v}_2 = \begin{pmatrix} 4 \\ 5 \\ 6 \end{pmatrix}, \quad
    \mathbf{v}_3 = \begin{pmatrix} 7 \\ 8 \\ 9 \end{pmatrix}
    $,
    as vectors of $\mathbb R^3$.
    \item 
    $
    p_1 = -1, \quad
    p_2 = x-1, \quad
    p_3 = (x-1)^2$,
    as vectors of $\mathcal P(\mathbb C)$.
    \item
    $
    \sigma_x=
    \begin{pmatrix} 0 & 1 \\ 1 & 0 \end{pmatrix}, \quad \sigma_y = \begin{pmatrix} 0 & -i \\ i & 0 \end{pmatrix}, \quad \sigma_z = \begin{pmatrix} 1 & 0 \\ 0 & -1 \end{pmatrix}
    $
    as vectors of $M_{2,2}(\mathbb C)$.
\end{enumerate}

\item 
A matrix is called \textit{anti-symmetric} if $A^\intercal = -A$.
Write down two (distinct!) bases in the
space of symmetric 2 × 2 matrices.
How many elements are in each basis?

\item 
A polynomial $p\in\mathcal P(\mathbb F)$ is \textit{even} if $p(x) = p(-x)$.
Prove that even polynomials form a subspace $U\subseteq \mathcal P(\mathbb F)$ and find a basis for even polynomials in $\mathcal P_7(\mathbb F)$.

\item
Let $U = 
\{
(z_1,z_2,z_3,z_4,z_5)\in\mathbb C^4
\mid
6z_1=z_2\text{ and }
z_3+2z_4=0
\}
$
Find a basis for $U$, then extend this basis to a basis of $\mathbb C^4$.

\item 
Let $U = \{p\in\mathcal P_4(\mathbb F)\mid p(6)=0\}$.
Find a basis for $U$, then extend this basis to $P_4(\mathbb F)$.

\item
Prove or give a counterexample:
if $v_1, v_2, v_3$ spans $V$, then the vectors
$w_1 = v_1 - v_2, w_2 = v_2 - v_3$ and $w_3 = v_3- v_1$ also span $V$.

\item 
Suppose that $v_1,\dots,v_n$ is linearly independent in $V$ and $w\in V$.
Prove that if $v_1+w,\dots,v_n+w$ is linearly dependent, then $w\in\text{span}(v_1,\dots,v_n)$.


\item 
Prove that $\mathbb F^\infty$ is infinite-dimensional.
Find a non-zero finite-dimensional subspace.

\item 
Prove that the space $C(\mathbb R)$ of continuous functions $f
:\mathbb R\to\mathbb R$ is infinite-dimensional.
\textit{Hint:} you know an infinite-dimensional subspace.

\item
Are the polynomials
\[
x^3-x^2+1 ,x^3-x^2+3, 5x^3-x^2+1, 17x^3-x^2+1\text{ and } x^2+6
\]
are linearly independent? \textit{Hint:} dimension.

Now go to \href{https://chat.openai.com/}{chatgpt.com} and ask ChatGPT if these polynomials are linearly independent.
It will probably get it wrong.
When it does, have a conversation with it, and see if you can get it to correct its mistakes.

\noindent
\textbf{Note:}
ChatGPT does not include a logic engine. It tries to answer math questions just by
pattern-matching the language, and it tends to agree with whatever you tell it.
\end{enumerate}

\end{document}