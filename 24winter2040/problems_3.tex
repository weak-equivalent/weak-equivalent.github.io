\documentclass[a4paper,11pt]{article}



\usepackage[utf8]{inputenc}
\usepackage[T1]{fontenc}
%\usepackage[portuguese]{babel}

%\usepackage[osf,sc]{mathpazo}
\usepackage[top=1in, bottom=1in, left=1.25in, right=1.25in]{geometry}

\usepackage{amsmath,amsthm,amssymb}
\usepackage{mathrsfs} % \mathsf

\usepackage[demo]{graphicx}
\usepackage[shortlabels]{enumitem}
\usepackage{tikz-cd}

\usepackage{quiver}
% surrounding boxes
\usepackage{tcolorbox}
\tcbuselibrary{breakable,skins}

\usepackage{hyperref}
\hypersetup{
    colorlinks,
    linkcolor={red!50!black},
    citecolor={blue!50!black},
    urlcolor={blue!80!black}
}

\title{Problem Set 3}
\author{Matrix Theory \& Linear Algebra II}
\date{}
% --new title page--



% \swapnumbers % 1.1.1. Teorema

\newtheorem{theorem}{Theorem}
\newtheorem{axiom}[theorem]{Axiom}
\newtheorem{claim}[theorem]{Claim}
\newtheorem{corollary}[theorem]{Corollary}
\newtheorem{lemma}[theorem]{Lemma}
\newtheorem{proposition}[theorem]{Proposition}


\theoremstyle{definition}
\newtheorem{remark}[theorem]{Remark}
\newtheorem{definition}[theorem]{Definition}
\newtheorem{example}[theorem]{Example}
\newtheorem{exercise}[theorem]{Exercício}
\newtheorem{fact}[theorem]{Fato}
\newtheorem{notation}[theorem]{Notation}



\newcommand{\bC}{\mathcal C}
\newcommand{\bF}{\mathcal F}
\newcommand{\bR}{\mathcal R}

\newcommand{\cP}{\mathcal P}

\newcommand{\dashline}{\noindent\rule{2cm}{0.4pt}}

\begin{document}
\maketitle
\thispagestyle{empty}

\noindent
Given a vector $v\in V$ and a basis $\mathcal B=\{e_1,\dots,e_n\}$ for $v$, we can find unique coefficient such that
\begin{equation}
v=a_1e_1+\cdots+a_ne_n.
\end{equation}
The numbers $a_1,\dots, a_n$ are the \textit{coordinates} of $v$ and depend on $\mathcal B$.
We also use the notation
\[
[v]_\mathcal B = \begin{bmatrix}
    a_1 \\ \vdots \\ a_n 
\end{bmatrix}_\mathcal B
\]

Given a linear transformation $T:V\to W$ and a basis $\mathcal A = \{f_1,\dots, f_m\}$ for $W$, the \textit{matrix of T} is the $m\times n$ matrix
\[
[T]_{\mathcal B,\mathcal A}
=
[[T(e_1)]_\mathcal A \mid \cdots \mid [T(e_1)]_\mathcal A ].
\]
The coordinates of the vector $T(v)$ can be found via matrix-vector multiplication:
\[
[T(v)]_\mathcal A
=
[T]_{\mathcal B,\mathcal A}[v]_\mathcal B
\]

\hrulefill 

\begin{enumerate}[(1)]
\item 
Read the handout available at \url{https://people.math.harvard.edu/~knill/teaching/math19b_2011/handouts/lecture08.pdf}, then do questions 1 and 2b.

\item 
Out of the fours functions from $M_{2,2}(\mathbb C)$ to $M_{2,2}(\mathbb C)$, two are linear transformations, and the other two are not.
For the ones that are, check the conditions for a linear transformation.
For the ones that are not, give a reason: explain one of the axioms for linear transformations that fails. (There might be more than one!)
\begin{enumerate}[(a)]
\item
The function $f:M_{2,2}(\mathbb C)\to M_{2,2}(\mathbb C)$ defined by $f(A) = A^\intercal$.
\item 
The function $f:M_{2,2}(\mathbb C)\to M_{2,2}(\mathbb C)$ defined by $f(A) = MAM^{-1}$, where $M$ is an invertible matrix.
\item 
The function $f:M_{2,2}(\mathbb C)\to M_{2,2}(\mathbb C)$ defined by $f(A) = A^2$.
\item 
The function $f:M_{2,2}(\mathbb C)\to M_{2,2}(\mathbb C)$ defined by $f(A) = A+I$.
\end{enumerate}

\item
  Let $\mathcal B = \left\{\begin{bmatrix} 2 \\ -1 \end{bmatrix},
    \begin{bmatrix} 3 \\ 2 \end{bmatrix}\right\}$ be a basis of $\mathbb R^2$  and let ${x} = \begin{bmatrix} 5 \\ -7 \end{bmatrix}$. Find $[x]_\mathcal B$.


\item
  Let $B = \left\{\begin{bmatrix} 1 \\ -1 \\ 2 \end{bmatrix},
    \begin{bmatrix} 2 \\ 1 \\ 2 \end{bmatrix},
    \begin{bmatrix} -1 \\ 0 \\ 2 \end{bmatrix}\right\}$ be a basis of $\mathbb R^3$ and let
  ${x} = \begin{bmatrix} 5 \\ -1 \\ 4 \end{bmatrix}$ be a
  vector in $\mathbb R^3$. Find $[x]_B$.


  


\item
  Let $T: \mathbb R^2 \to \mathbb R^2$ be a linear transformation defined by
  \begin{equation*}
    T \left(\begin{bmatrix} a \\ b \end{bmatrix}\right)
    = \begin{bmatrix}a+b \\ a-b \end{bmatrix}.
  \end{equation*}
  \begin{enumerate}[(a)]
  \item
  What is the null space of $T$? What is its range? 
  \item
  Consider the two bases
  \begin{equation*}
    B_1 = \left\{\begin{bmatrix} 1 \\ 0 \end{bmatrix},
      \begin{bmatrix} -1 \\ 1 \end{bmatrix}\right\}
  \text{ and } 
    B_2 = \left\{\begin{bmatrix} 1 \\ 1 \end{bmatrix},
      \begin{bmatrix} 1 \\ -1 \end{bmatrix}\right\}.
  \end{equation*}
  Find the matrix $M_{B_2,B_1}$ of $T$ with respect to the bases $B_1$ and $B_2$.
\end{enumerate}
\item
  Let $M=\begin{bmatrix}1 & 2 \\ -2 & 1 \end{bmatrix}$, and
  consider the linear transformation $T:M_{2,2}(\mathbb C)\to M_{2,2}(\mathbb C)$ given
  by $T(A) = MAM$. Find the matrix of $T$ with respect to the basis
  \begin{equation*}
    B=\left\{
      \begin{bmatrix} 1 & 0 \\ 0 & 0 \end{bmatrix},
      \begin{bmatrix} 0 & 1 \\ 0 & 0 \end{bmatrix},
      \begin{bmatrix} 0 & 0 \\ 1 & 0 \end{bmatrix},
      \begin{bmatrix} 0 & 0 \\ 0 & 1 \end{bmatrix}
    \right\}.
  \end{equation*}
  What is the null space of $T$? What is its range? 


\item
  Consider the linear transformation $T:\mathcal P_3(\mathbb F)\to\mathcal P_3(\mathbb F)$ given by
  $T(p(x)) = p(x+1)$. Find $[T]_{B,B}$, where $B=\{1,x,x^2,x^3\}$.

\item 
Describe the null spaces of the transformations defined in the previous three questions.
What is the dimension of their ranges?


\item
Recall that the $n$-th Taylor polynomial of a differentiable function $f:\mathbb R\to\mathbb R$, centered around a point $a\in \mathbb R$, is the polynomial
\[
p_n(x) = f(a)+f'(a)(x-a)+\frac{f''(a)}{2!}(x-a)^2+\cdots
\]
Let $g(x) = 2\sin(x-\pi)$.
\begin{enumerate}[(a)]
    \item
    Write down an expression for $p_2(x)$ of $g$ centered around $\pi$.
    \item
    What are the coordinates of $p_2(x)$ in the basis $\{1,x,x^2\}$?
    \item 
    What are the coordinates of $p_2(x)$ in the basis $\{1,(x-1),(x-1)^2\}$?    
\end{enumerate}
\item 
Define an integration linear transformation $\int:\mathcal P_n(\mathbb R)\to \mathcal P_{n+1}$ such that $\frac{d}{dx}\circ\int $ is the identity transformation on $\mathcal P_n$.

\end{enumerate}


\end{document}