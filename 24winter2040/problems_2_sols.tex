\documentclass[a4paper,11pt]{article}



\usepackage[utf8]{inputenc}
\usepackage[T1]{fontenc}
%\usepackage[portuguese]{babel}

%\usepackage[osf,sc]{mathpazo}
\usepackage[top=1in, bottom=1in, left=1.25in, right=1.25in]{geometry}

\usepackage{amsmath,amsthm,amssymb}
\usepackage{mathrsfs} % \mathsf

\usepackage[demo]{graphicx}
\usepackage[shortlabels]{enumitem}
\usepackage{tikz-cd}

\usepackage{quiver}
% surrounding boxes
\usepackage{tcolorbox}
\tcbuselibrary{breakable,skins}

\usepackage{hyperref}
\hypersetup{
    colorlinks,
    linkcolor={red!50!black},
    citecolor={blue!50!black},
    urlcolor={blue!80!black}
}

\title{Problem Set 2}
\author{Matrix Theory \& Linear Algebra II}
\date{}
% --new title page--


\setlength{\abovedisplayskip}{1pt}
\setlength{\belowdisplayskip}{1pt}

% \swapnumbers % 1.1.1. Teorema

\newtheorem{theorem}{Theorem}
\newtheorem{axiom}[theorem]{Axiom}
\newtheorem{claim}[theorem]{Claim}
\newtheorem{corollary}[theorem]{Corollary}
\newtheorem{lemma}[theorem]{Lemma}
\newtheorem{proposition}[theorem]{Proposition}


\theoremstyle{definition}
\newtheorem{remark}[theorem]{Remark}
\newtheorem{definition}[theorem]{Definition}
\newtheorem{example}[theorem]{Example}
\newtheorem{exercise}[theorem]{Exercício}
\newtheorem{fact}[theorem]{Fato}
\newtheorem{notation}[theorem]{Notation}


\newenvironment{solution}
  {\renewcommand\qedsymbol{$\blacksquare$}\begin{proof}[Solution]}
  {\end{proof}}


\newcommand{\bC}{\mathcal C}
\newcommand{\bF}{\mathcal F}
\newcommand{\bR}{\mathcal R}

\newcommand{\cP}{\mathcal P}

\newcommand{\dashline}{\noindent\rule{2cm}{0.4pt}}

\begin{document}
\maketitle
\thispagestyle{empty}

\noindent
In this problem set, $\mathbb F$ denotes either $\mathbb R$ or $\mathbb C$ (i.e. if $\mathbb F$ is in the question, then the solution should be agnostic to whether it was $\mathbb R$ or $\mathbb C$) and $V$ denotes a vector space over $\mathbb F$.

\begin{enumerate}[(1)]
\item
Which of the following lists are linearly independent?
\begin{enumerate}[(a)]
    \item 
    $
    \mathbf{v}_1 = \begin{pmatrix} 1 \\ 2 \\ 3 \end{pmatrix}, \quad
    \mathbf{v}_2 = \begin{pmatrix} 4 \\ 5 \\ 6 \end{pmatrix}, \quad
    \mathbf{v}_3 = \begin{pmatrix} 7 \\ 8 \\ 9 \end{pmatrix}
    $,
    as vectors of $\mathbb R^3$.
    \item 
    $
    p_1 = -1, \quad
    p_2 = x-1, \quad
    p_3 = (x-1)^2$,
    as vectors of $\mathcal P(\mathbb C)$.
    \item
    $
    \sigma_x=
    \begin{pmatrix} 0 & 1 \\ 1 & 0 \end{pmatrix}, \quad \sigma_y = \begin{pmatrix} 0 & -i \\ i & 0 \end{pmatrix}, \quad \sigma_z = \begin{pmatrix} 1 & 0 \\ 0 & -1 \end{pmatrix}
    $
    as vectors of $M_{2,2}(\mathbb C)$.
\end{enumerate}

\begin{solution}
\begin{enumerate}[(a)]
\item 
We have to check whether the equation
\[
x\begin{pmatrix} 1 \\ 2 \\ 3 \end{pmatrix}+
y\begin{pmatrix} 4 \\ 5 \\ 6 \end{pmatrix}+
z\begin{pmatrix} 7 \\ 8 \\ 9 \end{pmatrix}=0
\]
has any non-zero solutions, which correspond to solutions of the system
\[
\begin{cases}
x+4y+7z = 0\\
2x+5y+8z = 0\\
3x+6y+9z=0
\end{cases},
\]
which you can find however you'd like.
One solution suffices to prove linear dependence, here is an example:
\[
1\cdot\begin{pmatrix} 1 \\ 2 \\ 3 \end{pmatrix}
-2\cdot\begin{pmatrix} 4 \\ 5 \\ 6 \end{pmatrix}
+1\cdot\begin{pmatrix} 7 \\ 8 \\ 9 \end{pmatrix}=0.
\]
\item 
We have to check whether the equation
\[
a\cdot(-1)+b\cdot(x-1)+c\cdot (x-1)^2=0
\]
has no non-zero solutions.
Expanding out and collecting coefficients, the equation becomes
\[
(-a-b+c)\cdot 1 + (b-2c)\cdot x + c\cdot x^2=0.
\]
A polynomial is zero iff its coefficients are zero (in other words, $\{1,x,x^2\}$ is l.i.).
So we have the system
\[
\begin{cases}
    -a-b+c=0\\
    b-2c=0\\
    c=0
\end{cases}
\]
which clearly has only $a=b=c=0$ as a solution.
So the set is linearly independent.
\item 
We have to check whether the equation
\[
a\cdot\sigma_x+b\cdot\sigma_y+c\cdot \sigma_z=0
\]
has no non-zero solutions.
By expanding out coefficients, this corresponds to the equation
\[
\begin{pmatrix}
    c & a-i\cdot b\\ a+i\cdot b & c
\end{pmatrix}
=
\begin{pmatrix}
    0&0\\0&0
\end{pmatrix},
\]
so $c=0$ and $a+ib=a-ib=0 \implies a=b=0$.
So the set is linearly independent.

\begin{remark}The matrices $\sigma_x$, $\sigma_y$ and $\sigma_z$ are called \textit{Pauli matrices} and are fundamental in quantum mechanics.\end{remark}
\end{enumerate}
\end{solution}

\item 
A matrix is called \textit{anti-symmetric} if $A^\intercal = -A$.
Write down two (distinct!) bases in the
space of \textbf{anti-}\footnote{This was a typo in the original problem set.}symmetric 2 × 2 matrices.
How many elements are in each basis?
\begin{solution}
A matrix is anti-symmetric if
\[
\begin{bmatrix}
    a & b \\ c & d
\end{bmatrix}
=
\begin{bmatrix}
    -a & -d \\ -b & -d
\end{bmatrix}.
\]
In particular $a=-a$, so $a=0$.
Similarly $c=0$
Also $b=-d$.

Summarizing, our vector space is
\[V=\left\{
\begin{bmatrix}
    0& b \\ -b & 0
\end{bmatrix}
:
b\in\mathbb F\right\}.
\]
The matrix $A = \begin{bmatrix}
    0& 1 \\ -1 & 0
\end{bmatrix}$ spans $V$, and the set $\{A\}$ is trivially linear independent so it forms a basis for $V$ in fact.
Another basis is for instance $\left\{\begin{bmatrix}
    0& 10 \\ -10 & 0
\end{bmatrix}\right\}$.
\end{solution}

\item 
A polynomial $p\in\mathcal P(\mathbb F)$ is \textit{even} if $p(x) = p(-x)$.
Prove that even polynomials form a subspace $U\subseteq \mathcal P(\mathbb F)$ and find a basis for even polynomials in $\mathcal P_7(\mathbb F)$.
\begin{solution}
Even monic powers $1, x^2,x^4,\dots,x^{2n},\dots$ are even polynomials since
\[
(-x)^{2n} = (-1)^{2n}x^{2n} = x^{2n}.
\]
Odd powers $x,x^3,x^5,\dots,x^{2n+1},\dots$ are not:
\[
(-x)^{2n+1} = (-1)^{2n+1}x^{2n+1} = -x^{2n+1}\neq x^{2n+1}.
\]
So a polynomial is even if it is in the span of $1,x^2,x^4,\dots$, which is a linearly independent set so it is also a basis for the subspace of even polynomials.

In the case of $\mathcal P_7(\mathbb F)$, a basis for even polynomials is $\{1,x^2,x^4,x^6\}$.
\end{solution}

\item
Let\footnote{There was an extra $z_5$ in the posted problem set.} $U = 
\{
(z_1,z_2,z_3,z_4)\in\mathbb C^4
\mid
6z_1=z_2\text{ and }
z_3+2z_4=0
\}
$
Find a basis for $U$, then extend this basis to a basis of $\mathbb C^4$.
\begin{solution}
We can write $U$ as the set
\[
U = 
\{
(6a,a,-2b,b)\in\mathbb C^4
\mid
a,b\in\mathbb C
\}.
\]
Since we need two numbers to describe $U$, it is a good guess that it is two-dimensional.
Indeed, the set $\{(6,1,0,0),(0,0,-2,1)\}$ is linearly independent (it has just two vectors, which are not multiple of each other) and spans $U$, so it is a basis for $U$.

To extend to a basis of $U$ we can add the canonical basis and remove elements, or just add random vectors - you will usually get a set of linearly independent vectors.
For instance
$\{(6,1,0,0),(0,0,-2,1),(1,0,0,0),(0,0,1,0)\}$
is linearly independent (as you should check) and has 4 elements, so it is a basis for $\mathbb C^4$.
\end{solution}

\item 
Let $U = \{p\in\mathcal P_4(\mathbb F)\mid p(6)=0\}$.
Find a basis for $U$, then extend this basis to $P_4(\mathbb F)$.
\begin{solution}
We will borrow the observation that
\[
\mathcal B=\{
(x-6),(x-6)^2,(x-6)^3,(x-6)^4
\}
\]
is linearly independent set of vectors in $U$ (see question 1b), so the dimension of $U$ is at least 4.
Since $P_4(\mathbb F)$ is 5 dimensional and $1\notin U$, the dimension of $U$ is at most 4.
This implies that $\dim U = 4$ from which we conclude that $\mathcal B$ is a basis.

By adding back $1\in\mathcal P(\mathbb F)$ we get a basis for $\mathcal P(\mathbb F)$:
\[
\mathcal B=\{
1,(x-6),(x-6)^2,(x-6)^3,(x-6)^4
\}
\]
\end{solution}


\item
Prove or give a counterexample:
if $v_1, v_2, v_3$ spans $V$, then the vectors
$w_1 = v_1 - v_2, w_2 = v_2 - v_3$ and $w_3 = v_3- v_1$ also span $V$.
\begin{solution}
False, for instance $v_1=v_2=v_3 = 1\in\mathbb R$ spans $\mathbb R$ (seen as a one-dimensional real vector space), but $v_1-v_2 = v_2-v_3 = v_3-v_1 = 0$ does not.
\end{solution}
\item 
Suppose that $v_1,\dots,v_n$ is linearly independent in $V$ and $w\in V$.
Prove that if $v_1+w,\dots,v_n+w$ is linearly dependent, then $w\in\text{span}(v_1,\dots,v_n)$.
\begin{solution}
If $v_1+w,\dots,v_n+w$ are linearly dependent then there exists a linear combination
\[
c_1\cdot (v_1+w)+\dots+c_n\cdot (v_n+w)=0
\]
where some of the coefficients are non-zero.
Rearranging:
\[
c_1\cdot v_1 +\cdots+ c_n\cdot v_n = -(c_1+\cdots+c_n)w
\]
The number $-(c_1+\cdots+c_n)$ has to be non-zero:
if it was zero, we would have a non-zero linear combination of $v_1,\dots,v_n$, contradicting linear independence.
So we can write
\[
w = -\frac{1}{c_1+\cdots+c_n}(c_1\cdot v_1 +\cdots+ c_n\cdot v_n),
\]
showing that $w$ is in the span of $v_1,\dots,v_n$.
\end{solution}


\item 
Prove that $\mathbb F^\infty$ is infinite-dimensional.
Find a non-zero finite-dimensional subspace.
\begin{solution}
Let $a_i\in\mathbb F^\infty$ be the sequence with $1$ in the $i$-th entry and $0$ everywhere else.
For instance
\begin{align*}
a_1 = (1,0,0,\dots),\quad a_2=(0,1,0,0,\dots),\dots
\end{align*}
These vectors are linearly independent, indeed a linear combination
\[
c_1\cdot a_{k_1}+c_2\cdot a_{k_2}+\cdots +c_n\cdot a_{k_n}=(0,0,\dots)
\]
implies that $c_1=c_2=\dots=c_n=0$ by comparing coefficients.
So we found a linearly independent set with infinitely many elements, hence any basis has at least infinitely many elements.

\noindent\textit{Alternate proof:}
for each natural number $n$, there is a subspace $\widehat{\mathbb F^n}\subseteq \mathbb F^\infty$ given by lists with only zeroes after the $n$-th entry.
It's not hard to see that $\widehat{\mathbb F^n}$ is $n$-dimensional.
Hence the dimension of $\mathbb F^\infty$ is larger than any finite number $n$, and so itself can't be a finite number.
\end{solution}
\item 
Prove that the space $C(\mathbb R)$ of continuous functions $f
:\mathbb R\to\mathbb R$ is infinite-dimensional.
\textit{Hint:} you know an infinite-dimensional subspace.
\begin{solution}
The set of all polynomials forms an infinite-dimensional subspace.
\end{solution}
\item
Are the polynomials
\[
x^3-x^2+1 ,x^3-x^2+3, 5x^3-x^2+1, 17x^3-x^2+1\text{ and } x^2+6
\]
are linearly independent? \textit{Hint:} dimension.

Now go to \href{https://chat.openai.com/}{chatgpt.com} and ask ChatGPT if these polynomials are linearly independent.
It will probably get it wrong.
When it does, have a conversation with it, and see if you can get it to correct its mistakes.

\noindent
\textbf{Note:}
ChatGPT does not include a logic engine. It tries to answer math questions just by
pattern-matching the language, and it tends to agree with whatever you tell it.
\begin{solution}
The vector space $\mathcal P_3(\mathbb R)$ is 4-dimensional and these are 5 vectors, so the set is linearly dependent.
The point of the question was to talk to Chat GPT. (:
\end{solution}
\end{enumerate}
\end{document}